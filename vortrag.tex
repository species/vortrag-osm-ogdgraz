\documentclass{beamer} 
%\documentclass[handout]{beamer} 

% Michael Maier, 2014.
% CC-BY-SA 3.0 at

\usepackage[utf8]{inputenc}
\usepackage[ngerman]{babel}

\title{OpenStreetMap - Die freie Weltkarte} 
\author{Michael Maier \textless Michael.Maier@student.tugraz.at\textgreater} 
\date{27. November 2014} 

\usetheme{Antibes}

\hypersetup{colorlinks=true,urlcolor=blue,linkcolor=white}

%\usebackgroundtemplatei{
%\includegraphics[width=\paperwidth,
%height=0.8\paperheight]{mag_map.png}
%}

\begin{document}

%\maketitle

\begin{frame} 


\begin{figure}
  \centering
  \includegraphics[width=.5\textwidth]{mag_map.png}
\end{figure}

\begin{center}
\Huge{OpenStreetMap\\}
\end{center}

\begin{center}
\Large{\emph{Die freie Weltkarte}}
\end{center}

\end{frame}


% Zielgruppe
% * Politiker, um die 60 ohne jedes technische Wissen

% tolle Bilder herzeigen!
% * irgendein Zoo
% * 3D
% * Tolle Kartenstile:
%     * OSM-Fr?
%     * stamen watercolor
%     * pistemap
%     * bicycle map
%     * OpenSeaMap
%
% Dokumentation (nenne es doku und nicht Wiki → verwirrent) -OK
% wenn wo doku fehlt, selber mitschreiben?

% Vorteile
% * aktuell!
%
% * 100e verschiedene Kartenstile 
%   den eigenen leichtgemacht mit Tilemill
% * Datenversionierung! -OK
% * Language-Independent -OK
%   show http://toolserver.org/~osm/locale/ru.html
% * jetzt in 3D!
% switch2osm net vergessen! -OK
% HOT Team nicht vergessen! → zu geschichtliches

% * kurze OSM-Vorstellung, Geschichtliches, Motivation -OK
% * Technogie, Datenmodell, Lizenz -OK
% * OSM Nutzen: Rohdaten, Web-Dienste, Apps


\section{Einleitung}

\begin{frame}{Vorstellung}

  \begin{itemize}
    \item Michael Maier \textless \href{mailto:Michael.Maier@student.tugraz.at}{Michael.Maier@student.tugraz.at}\textgreater
    \item Student an der TU Graz (Telematik)
\vspace{0.3cm}
    \item OpenStreetMap als Hobby seit Juli 2010
    \item Leite den Grazer OSM-Stammtisch seit Mai 2011
    \begin{itemize}
        \item Vorträge und Workshops zum Thema OSM seit 2012
        \item Freiberuflich OSM-Aufträge und Consulting
    \end{itemize}
\vspace{0.3cm}
    \item Open Government Data Meetup Graz seit Beginn
    \begin{itemize}
        \item Betreibe \href{http://opendatagraz.at}{www.opendatagraz.at}
        \item Versioniere OGD auf github
    \end{itemize}
  \end{itemize}
\end{frame}



% Folien zu
% * kurze OSM-Vorstellung, Geschichtliches, Motivation
%  1. OSM-Vorstellung
  % was ist es
  % wer steckt dahinter?
% Geschichtliches
  % Gegründet ... steve
  % user-wachstum
% Motivation
  % gegründet, weil es keine freien Geodaten gab
  % Wunschtraum: eine DB weltweit

\section{OpenStreetMap}

\begin{frame}{Was ist OpenStreetMap}

\begin{itemize}
  \item OpenStreetMap (OSM) ist eine freie Weltkarte nach dem Wiki-Prinzip "`Wikipedia der Karten"'
    \begin{itemize}
      \item \emph{Eigentlich eine Geo-Datenbank}
    \end{itemize}
\pause
  \item Entsteht aus der Arbeit von \textgreater 1,8\,M Hobbykartografen "`\emph{Mapper}"'

\end{itemize}


 \begin{center}
 \includegraphics[width=5.5cm]{sotm.jpg}
 \end{center}

\end{frame}

%\begin{frame}{Wer steht hinter OpenStreetMap}
%
%  \begin{itemize}
%    \item OpenStreetMap Foundation (Server, Rechtliche Vertretung)
%      \pause
%    \item Mapper ($\sim$20.000 aktiv), meist ohne Geo-Hintergrund
%    \begin{itemize}
%      \item Jährliche Konferenz - "`State of the Map"', heuer: Buenos Aires
%    \end{itemize}
%      \pause
%    \item Universitäten
%    \begin{itemize}
%      \item Bakk-, Master- und Doktorarbeiten mit OSM
%      \item Server-Hosting
%    \end{itemize}
%      \pause
%    \item Organisationen, die Daten sponsern
%    \begin{itemize}
%      \item Firmen wie Yahoo/Bing, die Luftbilder zur Verfügung stellen
%      \item Regierungen mit besseren Open-Data-Gesetzen als Österreich %!!!!!!!!!!!!!!!!!!!!FIXME
%  % BSP TIGER, USA
%  % Dänemark, Hausnummern
%  % Frankreich,Tschechien: Kataster
%    \end{itemize}
%      \pause
%    \item Firmen die mit OSM arbeiten, z.B.:
%    \begin{itemize}
%      \item Geofabrik (de)
%      \item MapQuest (us)
%      \item BikeCityGuide (Graz)
%    \end{itemize}
%  \end{itemize}
%
%\end{frame}

  
%{
% \usebackgroundtemplate{\includegraphics[height=10cm]{Osmdbstats2_users.png}}
%
%\begin{frame}{Geschichte von OpenStreetMap}
%  \vspace{0.6cm}
%\begin{itemize}
%  \item Start des Projekts im August 2004 durch \emph{Steve Coast}
%  \item Dezember 2006 - Yahoo erlaubt abzeichnen
%  \item Juli 2007 - Erste Konferenz, "`State Of The Map"'
%  \item August 2007 - 10.000 Registrierte Benutzer
%  \item März 2009 - 100.000 Registrierte Benutzer
%  \item Januar 2010 - Haiti--Projekt
%  \item November 2010 - Bing erlaubt abzeichnen
%  \item Juli 2011 - Erste "`State Of The Map Europe"' in Wien
%  \item Januar 2013 - 1.000.000 Registrierte Benutzer
%  \item Gestern - 1.491.901 Registrierte Benutzer
%\end{itemize}
%
%\end{frame}
%}




\begin{frame}{Warum OpenStreetMap?}

\hspace{0.5cm}Es beginnt 2004 mit einer Geschichte: 
  \vspace{0.3cm}

Ein Student ärgert sich, dass es in UK keine freien Geodaten gibt. 
  \vspace{0.3cm}

\parbox{9.5cm}{Die Daten auf streetmap.co.uk wurden mit Steuergeldern erstellt, man kann die Rohdaten jedoch nicht frei verwenden.}
\hfill
\raisebox{\dimexpr-\height+\baselineskip}{\includegraphics[height=1cm]{traurig.png}}

  \vspace{0.6cm}
\pause

Warum muss man für etwas, was bereits von der Allgemeinheit mit Steuergeld bezahlt wurde, nocheinmal bezahlen?
  \vspace{0.3cm}

\parbox{9.1cm}{Und darf es selbst dann nicht frei Nutzen? \\Doppelbesteuerung ist zumindest bei uns verboten?}
\hfill 
\raisebox{\dimexpr-\height+\baselineskip}{\includegraphics[height=1cm]{grantig.png}}

\pause

 \parbox{7.5cm}{\vspace{0.4cm}\hspace{0.5cm}$\Longrightarrow$ \hspace{0.5cm}Er gründet OpenStreetMap!} 
\raisebox{\dimexpr-\height+\baselineskip}{\includegraphics[height=1.2cm]{laugh.png}}

\end{frame}

%\begin{frame}{Vision einer besseren Welt}
%
% Sollte es nicht so sein:
%  \begin{itemize}
%    \item Es gibt weltweit EIN Portal für ALLE Verwaltungs-Daten 
%    \item In einheitlichem Format und Sprache
%    \item Alte Versionen verfügbar (Um Änderungen zu verfolgen)
%    \item Unkompliziert Fehler melden, oder selbst ausbessern kann
%\pause
%    \item Man keine Anträge stellen muss, sondern einfach einen Ausschnitt wählt und Rohdaten runterlädt
%    \item Alle Daten unter einen freien Lizenz nutzen kann
%  \end{itemize}
%
%  \begin{columns}[c]
%        \column{.5\textwidth}
%        \begin{center}
%  \includegraphics[width=3.5cm]{marble.png}
%  \end{center}
%        \column{.5\textwidth}
%      \begin{center}
%    \includegraphics[width=2.5cm]{cc-by-sa.pdf}
%  \end{center}
%\end{columns}
%
%\end{frame}


\section{Wie funktioniert OpenStreetMap?}

\begin{frame}{Woher kommen unsere Daten?}

\begin{itemize}
  \item Ursprünglich: GPS-Tracks
  \item Freiwillige tragen ihr Wissen bei: Jeder weiß viel über seine Umgebung:
	\begin{itemize}
	  \item Hausnummern, Straßennamen,
	  \item Restaurants, Bars, POIs, \dots
  \end{itemize}
  \pause
  \item Bei Mapping-Parties werden \\ gezielt Gebiete verbessert
\end{itemize}

  \vspace{0.4cm}
 99\% Handarbeit!

  \vspace*{-2.9cm}
 \hfill \includegraphics[width=4.2cm]{alps_mp.jpg}


  \pause
\begin{itemize}
  \item Unterstützt durch Open Government Data
  \begin{itemize}
    \item USA, TIGER Data (2008)
    \item Dänemark, Hausnummern (laufend synchronisiert)
    \item Graz, Steiermark, Wien, Engerwitzdorf... und 20 weitere
  \end{itemize}
\end{itemize}

\end{frame}

\section{OGD-Nutzung in OpenStreetMap}

\begin{frame}{OGD Graz in OpenStreetMap}
    
  \vspace*{-1.9cm}
 \hfill \includegraphics[width=3.0cm]{basiskarte.png}

  \vspace*{-2.0cm}
    Beispiel: Hausnummern

    \begin{itemize}
        \item Die Hausnummern sind auf der Grazer \\ Basiskarte 'analog' veröffentlicht
        \item Wir haben unsere fehlenden 35.000 \\mithilfe  der Grazer Basiskarte ergänzt
  \vspace*{0.2cm}
        \item Damit wurden die Grazer Hausnummern weltweit durchsuchbar
        \item Werden in Navigationssystemen, zB BikeCityGuide genutzt
    \end{itemize}

  \vspace*{-0.7cm}
\begin{columns}[c]
\column{.5\textwidth}
\begin{center}
\includegraphics[width=4.5cm]{bcg.png}
\end{center}
\column{.5\textwidth}
\begin{center}
\includegraphics[width=5.5cm]{osm-gg3.png}
\end{center}
\end{columns}

\end{frame}
\begin{frame}{OGD Verbundlinie in OpenStreetMap}
    Beispiel: Haltestellen der Verbundlinie

    \begin{itemize}
        \item Haltestellen des Verkehrsverbundes Steiermark \\
            werden derzeit in die OpenStreetMap importiert
    \begin{itemize}
        \item $~$8.000/$~$16.000 schon importiert
    \end{itemize}
        \item Die nun öffentlichen Referenz- \\ nummern, z.B. "ref:IFOPT=\\at:46:4226:1:aus" können \\ für intermodales Routing \\ genutzt werden
  \vspace*{0.2cm}
        \item Bus/Bahn/Bim-Linien werden \\ importiert, daraus z.B. die \\ \href{http://ÖPNVkarte.de}{ÖPNVkarte.de} erstellt
  \vspace*{-4.3cm}
    \end{itemize}
 \hfill \includegraphics[width=5.0cm]{oepnv.png}

\end{frame}

\begin{frame}{OGD zur Qualitätssicherung in OpenStreetMap}
  Beispiel: Qualitätssicherung:
  \vspace*{0.2cm}

  Viele OGD-Veröffentlichungen werden zur Qualitätssicherung in der OpenStreetMap genutzt:
  \begin{itemize}
                  \item Apotheken und Krankenanstalten
                  \item Bibliotheken und Bildungsstandorte
                  \item Kinder- und Jugendorganisationen
                  \item Öffentliche Brunnen
              \end{itemize}
   Abgleich wurde in beide Richtungen durchgeführt:
  \begin{itemize}
                  \item Fahrradständer
                  \item Straßenverzeichnis
              \end{itemize}
                  
              Hier konnten als "`Rückfluss"' aus der Community fehlende / fehlerhafte Daten der Stadt Graz korrigiert werden

\end{frame}

\begin{frame}{OGD Visualisierung: Baumkataster}
    \href{http://www.opendatagraz.at/2014/06/23/grazer-baume/}{www.opendatagraz.at/2014/06/23/grazer-baume/}
    \includegraphics[width=10.0cm]{visualisierung-tyr.png}
\end{frame}
% 100% freie Software
% → jeder kann den Software-Stack verwenden http://openaviationmap.org/
% Quality Assurance
% * Es gibt automatische Q/A-Tools
% * kaum Streitfälle - wenn dann Mailinglist, Data Working Group
% * 
% tolle Bilder herzeigen!
% * irgendein Zoo
% * 3D -FIXME
% * Tolle Kartenstile:
%     * OSM-Fr?
%     * stamen watercolor
%     * pistemap
%     * bicycle map
%     * OpenSeaMap

% Wie daraus Karten generieren:
% 1F Toolchain
% nF Beispiele:
% * irgendein Zoo Mapnik -OK
% * 3D
% * Tolle Kartenstile:
%     * OSM-Fr? -OK http://tile.openstreetmap.fr/ -OK
%     * stamen watercolor -OK http://maps.stamen.com/watercolor/ -OK
%     * pistemap /snow http://www.opensnowmap.org/ -OK
%     * bicycle map -OK http://cyclemap.org/ -OK
%     * OpenSeaMap -OK http://openseamap.org/ -OOK
%   show http://toolserver.org/~osm/locale/ru.html
    

%
%\begin{frame}{Freie Datendaten ermöglichen freie Kartenstile}
%
%Der Standard-Kartenstil (Mapnik) ist auf \href{https://github.com/gravitystorm/openstreetmap-carto/}{Github} verfügbar
%\begin{itemize}
%  \item Kann für persönlichen Stil angepasst werden
%  \item Er wird kollektiv weiterentwickelt, jeder kann mitmachen!
%\end{itemize}
%
%\begin{center}
%\includegraphics[width=7cm]{style-mapnik.png}
%\end{center}
%
%\vspace{-0.5cm}
%weitere Stile: \url{http://wiki.osm.org/Featured\_tiles}
%
%\end{frame}
%
%\hypersetup{urlcolor=cyan}
%
%\begin{frame}{Französischer Stil:\hfill\url{http://tile.openstreetmap.fr/}}
%\begin{center}
%\includegraphics[height=7cm]{style-french.png}
%\end{center}
%\end{frame}
%
%\begin{frame}{Stamen Watercolor:\hfill\href{http://maps.stamen.com/watercolor/}{http://maps.stamen.com/watercolor}}
%\begin{center}
%\includegraphics[height=7cm]{style-stamen.png}
%\end{center}
%\end{frame}
%\hypersetup{urlcolor=blue}
%
%
%\hypersetup{urlcolor=cyan}
%
%\begin{frame}{Ski-Karte :\hfill\url{http://www.opensnowmap.org/}}
%\begin{center}
%\includegraphics[height=6.1cm]{style-snow.png}
%\end{center}
%\end{frame}
%
%\begin{frame}{See-Karte :\hfill\url{http://www.openseamap.org/}}
%\begin{center}
%\includegraphics[height=7cm]{style-seamap.png}
%\end{center}
%\end{frame}
%
%\begin{frame}{Fahrrad-Karte :\hfill\url{http://www.opencyclemap.org/}}
%\begin{center}
%\includegraphics[height=6cm]{style-cycle.png}
%\end{center}
%\end{frame}
%
%
%
%
%
%\section{OpenStreetMap Nutzen}
%
%
%\begin{frame}{Die Zukunft... 3D! }
%  Neu! Jetzt auch in 3D! Beispielsweise auf \href{http://maps.osm2world.org/?zoom=17&lat=47.06156&lon=15.46983&layers=BF0FTFFF}{maps.osm2world.org}.
%
%  \includegraphics[width=0.9\textwidth]{3d.png}
%
%
%\end{frame}
%
\begin{frame}{Hilfe}
Fragen? 
\begin{itemize}
  \item Dokumentation: \href{http://wiki.openstreetmap.org}{wiki.openstreetmap.org}
  \begin{itemize} 
    \item Mitmachen? \href{http://learnosm.org/}{learnosm.org}
  \end{itemize}
  \item Fragen stellen? \\ $\Rightarrow$ Mailingliste \href{http://lists.openstreetmap.org/listinfo/talk-at}{talk-at}
\vspace{1cm}
  \item Grazer \href{https://wiki.openstreetmap.org/wiki/Graz/Stammtisch}{Stammtisch}
  \begin{itemize}
      \item jeden 2. Montag im Monat 
      \item Brot $\&$ Spiele
  \end{itemize}

\end{itemize}

 \vspace*{-2.0cm}
\hfill\includegraphics[width=5cm]{Osm_graz_members_2011.jpeg}

\end{frame}

\section{Ende}

\begin{frame}{Vielen Dank für die Aufmerksamkeit!}

  Folien zur OGD-Strategiegruppensitzung, 27.11.2014, Graz
\vspace{1cm}

Erstellt mittels \LaTeX Beamer, Quelltext: \href{https://github.com/species/vortrag-osm-ogdgraz}{Github}.
\vspace{1cm}

\href{mailto:michael.maier@student.tugraz.at}{Michael Maier}

Twitter: \href{https://twitter.com/osmgraz}{@osmgraz}
\vspace{1cm}

Folien unter: \includegraphics[width=1cm]{cc-by-sa.pdf}. 

Alle Daten ODbL, OpenStreetMap Contributors.

\end{frame}

\end{document}
